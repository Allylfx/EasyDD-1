\chapter{Talks}
	\section{Durham July 12--14 2017}
		Bridging the gap between the microscopic and macroscopic world is an on-going challenge for science and technology. If we hope to understand complex emergent phenomena we need to study systems that blur the line between micro and macro. In materials science, one such area is the study of extended defects called dislocations; whose nucleation and movement mediate the permanent deformation of materials. These large, high energy defects present very complex and long ranged interactions with each other, crystal boundaries, impurities, free surfaces, and themselves. As such, their dynamics are difficult to study experimentally. So we make justified assumptions and simplifications and create models that let us study them in detail. However if our models are to prove useful in real applications, they must be continually refined and improved by weakening assumptions and removing simplifications. Unfortunately, with increased refinement comes increased computational cost and new challenges. Therefore, finding faster alternatives that do not sacrifice accuracy is of the utmost importance---better yet if the alternatives are more accurate or exact. With the advent of increasingly accessible graphics processor units (GPUs) typically used in video gaming, the power of parallel processing is no longer exclusive to researchers with access to supercomputers. In this talk we will discuss the GPU implementation of exact solutions for the forces dislocations exert on the surfaces of materials.